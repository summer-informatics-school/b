\documentclass[a4paper,12pt]{article}

\usepackage[utf8]{inputenc}
\usepackage[T2A]{fontenc}
\usepackage[english,russian]{babel}

\usepackage{graphicx}
\usepackage{color}
\usepackage{indentfirst}
\usepackage{fancyhdr}
\usepackage{needspace}

\pagestyle{fancy}
\lhead{Зачет}
\rhead{ЛКШ.B}
\rfoot{2015, август}

\begin{document}
  \section{Графы}

    \subsection{Поиск в глубину}

      \subsubsection{Понятия и алгоритм}
        \begin{enumerate}
          \item Поиск (обход) графа
          \item Основной алгоритм
          \item Метки посещения
          \item Дерево обхода
          \item Классификация рёбер
          \item Время входа и выхода
          \item Метки трёх цветов
        \end{enumerate}

      \subsubsection{Применения}
        \begin{enumerate}
          \item Проверка на предка
          \item Компоненты связности
          \item Поиск циклов
          \item Топологическая сортировка
          \item Раскраска графа в два цвета
          \item Компоненты сильной связности
          \item Мосты
          \item Точки сочленения
          \item Компоненты двусвязности
          \item * Эйлеров цикл
          \item * Максимальное паросочетание в двудольном графе
          \item * Наименьший общий предок
        \end{enumerate}
 
    \subsection{Остовы}
      \begin{enumerate}
        \item Определение
        \item Лемма о разрезе
        \item Алгоритм Прима
        \item Алгоритм Краскала
      \end{enumerate}

    \subsection{Кратчайшие пути}
      \begin{enumerate}
        \item Алгоритм Дейкстры
        \item Алгоритм Дейкстры с приоритетной очередью
        \item Алгоритм Форда-Беллмана
        \item Нахождение кратчайшего пути с учетом отрицательных циклов
        \item Алгоритм Флойда
        \item Восстановление пути
        \item Поиск в ширину для графов с весами $\{0, 1\}$, ${0, \ldots, k}$
      \end{enumerate}

  \section{Структуры данных}

    \subsection{Дерево отрезков}
      \begin{enumerate}
        \item RMQ, RSQ, решения offline
        \item Общая концепция дерева отрезков
        \item Операция обновления элемента
        \item Операция запроса на отрезке
        \item Детали реализации
        \item Групповые операции
        \item Операция изменения на отрезке
      \end{enumerate}

    \subsection{Разреженные таблицы}

    \subsection{СНМ}
      \begin{enumerate}
        \item Ранговая эвристика
        \item Эвристика переподвешивания к корню
      \end{enumerate}

    \subsection{Декартово дерево}
      \begin{enumerate}
        \item Определение BST
        \item Операции в BST: поиск, вставка, удаление
        \item Операции в декартовом дереве
        \item Указатели
        \item Реализация
        \item Хранение размеров поддеревьев
        \item $k$-я статистика
        \item Запросы на отрезке значений: \texttt{count}, \texttt{sum}
        \item Групповые операции
        \item Неявный ключ
        \item Хранение массива в декартовом дереве, разрезания, обращения по индексу
        \item Групповые обновления: \texttt{add}, \texttt{reverse}
      \end{enumerate}

  \section{Динамическое программирование}
    \begin{enumerate}
      \item Общие принципы
      \item ДП на префиксах
        \begin{enumerate}
          \item Кратчайший путь в ациклическом графе
          \item Наибольшая возрастающая подпоследовательность
          \item Наибольшая общая подпоследовательность
        \end{enumerate}
      \item ДП по цифрам числа
        \begin{enumerate}
          \item Количество чисел от $0$ до $n$ с суммой цифр $k$
        \end{enumerate}
      \item ДП на подотрезках
        \begin{enumerate}
          \item Наибольшая подпоследовательность-палиндром
          \item Триангуляция многоугольника с максимальной суммой сторон
        \end{enumerate}
      \item ДП на поддеревьях
        \begin{enumerate}
          \item Взвешенное паросочетание в дереве
          \item Выделение поддерева размера $k$ с минимальныи количеством отрезанных ребер
          \item Задача о количестве вершин на расстоянии $k$ для каждой вершины
        \end{enumerate}
      \item ДП на подмножествах
        \begin{enumerate}
          \item Гамильтонов путь, цикл
          \item Минимальное число цветов, в которое можно раскрасить граф, перебор подмасок за $3^n$
        \end{enumerate}
      \item ДП по профилю
        \begin{enumerate}
          \item Замощение доминошками
        \end{enumerate}
    \end{enumerate}

  \section{Геометрия}

  \section{Строки}
    \begin{enumerate}
      \item Префикс-функция, алгоритм Кнута-Морриса-Пратта
      \item Z-функция
      \item Полиномиальный хеш
        \begin{enumerate}
          \item Реализация
          \item Сравнение строк
          \item Максимальный подпалиндром
        \end{enumerate}
      \item Бор
        \begin{enumerate}
          \item Реализация
          \item Сортировка
          \item Число различных подстрок
          \item Беспрефиксность кода
        \end{enumerate}
    \end{enumerate}

  \section{Комбинаторика}
    \begin{enumerate}
      \item Формулы для перестановок и сочетаний
      \item Связь сочетаний с биномом
      \item Динамика для подсчёта объектов
      \item Рекурсивная генерация комбинаторных объектов 
      \item Получение объекта по номеру 
      \item Получение номера по объекту 
      \item Генерация следующий/предыдущий
    \end{enumerate}

\end{document}
