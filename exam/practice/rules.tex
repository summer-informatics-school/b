\documentclass[a4paper,12pt]{article}

\usepackage[utf8]{inputenc}
\usepackage[T2A]{fontenc}
\usepackage[english,russian]{babel}

\usepackage{graphicx}
\usepackage{indentfirst}
\usepackage{fancyhdr}

\pagestyle{fancy}
\lhead{Практический зачёт}
\rhead{ЛКШ.2014.Июль.B}

\begin{document}

  \section{Зачёт}

  \subsection{Практический зачёт}

  Вещественная оценка за практический зачёт вычисляется по формуле:

  $$ R = \alpha \min_{k}{A_k} + \beta \frac{\sum_k{A_k}}{k} $$

  Здесь $A_k$~--- <<предварительная оценка>> за данную тему.
  $A_k = 0.5 \frac{P^s_k}{P_k} + 0.5 \frac{E^s_k}{E_k}$, где:
  \begin{itemize}
    \item $P_k$~--- количество задач на практических занятиях
      на тему $k$,
    \item $P^s_k$~--- количество решённых задач на практических
      занятиях на тему $k$,
    \item $E_k$~--- количество задач на практическом зачёте
      на тему $k$,
    \item $E^s_k$~--- количество решённых задач на практическом
      зачёте на тему $k$.
  \end{itemize}

  В программе выделяется пять тем: <<Графы>>, <<Структуры данных>>,
  <<Динамическое программирование>>, <<Геометрия>>.

  Коэффициенты $\alpha$ и $\beta$ подобраны так, чтобы стимулировать
  всестороннее развитие. А именно, $\alpha = 1.5$, $\beta = 4.5$.

  Например, школьник, не решивший ни одной задачи по одной из четырёх
  тем, получит заведомо не больше $3.375$. Школьник, равномерно
  разобравшийся
  со всеми темами, будет оценен несколько выше, чем школьник,
  потративший всё время на одну.

  Оценка получается по вещественной оценке $R$ следующим образом:

  \begin{center}
  \begin{tabular}{|c|c|}
    \hline
    $R$ & оценка \\
    \hline
    $\left[0.0, 2.5\right)$ & 2 \\
    \hline
    $\left[2.5, 3.0\right)$ & 3- \\
    \hline
    $\left[3.0, 3.5\right)$ & 3 \\
    \hline
    $\left[3.5, 3.75\right)$ & 3+ \\
    \hline
    $\left[3.75, 4.0\right)$ & 4- \\
    \hline
    $\left[4.0, 4.5\right)$ & 4 \\
    \hline
    $\left[4.5, 4.75\right)$ & 4+ \\
    \hline
    $\left[4.75, 5.0\right)$ & 5- \\
    \hline
    $\left[5.0, 5.5\right)$ & 5 \\
    \hline
    $\left[5.5, 6.0\right]$ & 5+ \\
    \hline
  \end{tabular}
  \end{center}

\end{document}
