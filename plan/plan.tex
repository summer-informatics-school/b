\documentclass[a4paper,12pt]{article}

\usepackage[utf8]{inputenc}
\usepackage[T2A]{fontenc}
\usepackage[english,russian]{babel}

\usepackage{graphicx}
\usepackage{color}
\usepackage{indentfirst}
\usepackage{fancyhdr}
\usepackage{needspace}

\pagestyle{fancy}
\lhead{План лекций}
\rhead{ЛКШ.2014.Июль.B}

\begin{document}
  \section{День 01. Поиск в глубину}

    \subsection{Понятия и алгоритм}
      \subsubsection{Поиск (обход) графа}
      \subsubsection{Основной алгоритм}
      \subsubsection{Метки посещения}
      \subsubsection{Дерево обхода}
      \subsubsection{Классификация рёбер}
      \subsubsection{Время входа и выхода}
      \subsubsection{Метки трёх цветов}
      \subsubsection{* Особенности реализации для дерева}

    \subsection{Применения}
      \subsubsection{Проверка на предка}
      \subsubsection{Компоненты связности}
      \subsubsection{Поиск циклов}
      \subsubsection{Топологическая сортировка}
      \subsubsection{Раскраска графа в два цвета}
      \subsubsection{Компоненты сильной связности}
      \subsubsection{Мосты}
      \subsubsection{Точки сочленения}
      \subsubsection{Компоненты двусвязности}
      \subsubsection{* Эйлеров цикл}
      \subsubsection{* Максимальное паросочетание в двудольном графе}
      \subsubsection{* Наименьший общий предок}

  \newpage

  \section{День 02. Остовы}

    \subsection{Определение}

    \subsection{Важное утверждение}

    \subsection{Алгоритм Прима}

    \subsection{Алгоритм Краскала}

    \subsection{СНМ}
      \subsubsection{Ранговая эвристика}
      \subsubsection{Эвристика переподвешивания к корню}

  \newpage

   \section{День 03. Динамическое программирование}

      \subsection{Общие принципы}

      \subsection{ДП на префиксах}
        \subsubsection{Кратчайший путь в ациклическом графе}
        \subsubsection{НВП}
        \subsubsection{НОП}

      \subsection{ДП по цифрам числа}
        \subsubsection{Количество чисел от $0$ до $n$ с суммой цифр $k$}

      \subsection{ДП на подотрезках}
        \subsubsection{Оптимальное перемножение матриц}
        \subsubsection{Наибольшая подпоследовательность-палиндром,
          подматрица-палиндром}
        \subsubsection{<<Казино>>}

      \subsection{ДП на поддеревьях}
        \subsubsection{Взвешенное паросочетание в дереве}
        \subsubsection{Выделение поддерева размера $k$}
        \subsubsection{Задача о разделении дерева}

  \newpage

  \section{День 04. Динамическое программирование}

    \subsection{Разбор}
      \subsubsection{<<Транзисторы>>}
      \subsubsection{<<Казино>>}
      \subsubsection{Выделение поддерева размера $k$}

    \subsection{ДП на подмножествах}
      \subsubsection{Гамильтонов путь, цикл}
      \subsubsection{Паросочетание в произвольном графе}

    \subsection{ДП по профилю}
      \subsubsection{Замощение доминошками}

  \newpage

  \section{День 05. Комбинаторика}

    \subsection{Генерация комбинаторных объектов}
      \subsubsection{*Перестановки}
      \subsubsection{*Разбиение на слагаемые}
      \subsubsection{*Сочетания}
      \subsubsection{*Скобочная последовательность}

    \subsection{Подсчёт объектов}

    \subsection{Получение по номеру}

    \subsection{Получение номера по объекту}

    \subsection{Генерация следующий/предыдущий}

  \newpage

  \section{День 06. Дерево отрезков}

    \subsection{RMQ, RSQ, решения offline}

    \subsection{Общая концепция дерева отрезков}

    \subsection{Операция обновления элемента}

    \subsection{Операция запроса на отрезке}

    \subsection{Детали реализации}

    \subsection{Примеры}
      \subsubsection{Количество инверсий}
      \subsubsection{<<Stars>>}

    \subsection{Групповые операции}

    \subsection{Операция изменения на отрезке}

    \subsection{* Разреженные таблицы}

\end{document}
