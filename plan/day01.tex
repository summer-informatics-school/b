\documentclass[a4paper,12pt]{article}

\usepackage[utf8]{inputenc}
\usepackage[T2A]{fontenc}
\usepackage[english,russian]{babel}

\usepackage{graphicx}
\usepackage{color}
\usepackage{indentfirst}
\usepackage{fancyhdr}
\usepackage{needspace}

\pagestyle{fancy}
\lhead{План лекций}
\rhead{ЛКШ.2014.Июль.B}

\begin{document}
  \section{День 01. Поиск в глубину}

    \subsection{Понятия и алгоритм}
      \subsubsection{Поиск (обход) графа}
      \subsubsection{Основной алгоритм}
      \subsubsection{Метки посещения}
      \subsubsection{Дерево обхода}
      \subsubsection{Классификация рёбер}
      \subsubsection{Время входа и выхода}
      \subsubsection{Метки трёх цветов}
      \subsubsection{* Особенности реализации для дерева}

    \subsection{Применения}
      \subsubsection{Проверка на предка}
      \subsubsection{Компоненты связности}
      \subsubsection{Поиск циклов}
      \subsubsection{Топологическая сортировка}
      \subsubsection{Раскраска графа в два цвета}
      \subsubsection{Компоненты сильной связности}
      \subsubsection{Мосты}
      \subsubsection{Точки сочленения}
      \subsubsection{Компоненты двусвязности}
      \subsubsection{* Эйлеров цикл}
      \subsubsection{* Максимальное паросочетание в двудольном графе}
      \subsubsection{* Наименьший общий предок}

\end{document}
