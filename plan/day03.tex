\documentclass[a4paper,12pt]{article}

\usepackage[utf8]{inputenc}
\usepackage[T2A]{fontenc}
\usepackage[english,russian]{babel}

\usepackage{graphicx}
\usepackage{color}
\usepackage{indentfirst}
\usepackage{fancyhdr}
\usepackage{needspace}

\pagestyle{fancy}
\lhead{План лекций}
\rhead{ЛКШ.2014.Июль.B}

\begin{document}
   \section{День 03. Динамическое программирование}

      \subsection{Общие принципы}

      \subsection{ДП на префиксах}
        \subsubsection{Кратчайший путь в ациклическом графе}
        \subsubsection{НВП}
        \subsubsection{НОП}

      \subsection{ДП по цифрам числа}
        \subsubsection{Количество чисел от $0$ до $n$ с суммой цифр $k$}

      \subsection{ДП на подотрезках}
        \subsubsection{Оптимальное перемножение матриц}
        \subsubsection{Наибольшая подпоследовательность-палиндром,
          подматрица-палиндром}
        \subsubsection{<<Казино>>}

      \subsection{ДП на поддеревьях}
        \subsubsection{Взвешенное паросочетание в дереве}
        \subsubsection{Выделение поддерева размера $k$}
        \subsubsection{Задача о разделении дерева}
\end{document}
