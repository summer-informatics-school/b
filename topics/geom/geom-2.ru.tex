\documentclass[a4paper,12pt]{article}

\usepackage[utf8]{inputenc}
\usepackage[T2A]{fontenc}
\usepackage[english,russian]{babel}

\usepackage{graphicx}
\usepackage{color}
\usepackage{indentfirst}
\usepackage{fancyhdr}
\usepackage{needspace}
\usepackage{listings}
\usepackage{tikz}

\renewcommand{\le}{\leqslant}
\newcommand{\ora}{\overrightarrow}
\def\v{\vec}
\def\vv{\ora}

\pagestyle{fancy}
\lhead{Геометрия: \textrm{II}~часть}
\rhead{ЛКШ.2015.Июль.B}

\usepackage{amsmath, amssymb}
\DeclareMathOperator{\pr}{pr}

\begin{document}
\setcounter{tocdepth}{7}
%\renewcommand\cftchapafterpnum{\vskip{10pt}}
%\renewcommand\cftsecafterpnum{\vskip{15pt}}
\tableofcontents
\newpage

  \section{Многоугольники}
    \subsection{Площадь многоугольника}
    \subsection{Проверка принадлежности точки многоугольнику}
      \subsubsection{Общий случай, $O(n)$}
      \subsubsection{Выпуклый многоугольник, $O(\log n)$}
    \subsection{Выпуклая оболочка}
  \section{Окружности}
    \subsection{Пересечение прямой и окужности}
    \subsection{Пересечение двух окужностей}
    \subsection{Касательная к окружности из точки}
      \subsubsection{Количество касательных в зависимости от положения}
      \subsubsection{Построение}
    \subsection{Общая касательная к двум окружностям}
      \subsubsection{Количество касательных, внутренние и внешние касательные}
      \subsubsection{Построение (сжатие, расжатие окружностей)}
  \section{* Минимальное/максимальное расстояния в множестве точек}
\end{document}
