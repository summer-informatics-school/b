\documentclass[a4paper,12pt]{article}

\usepackage[utf8]{inputenc}
\usepackage[T2A]{fontenc}
\usepackage[russian,english]{babel}

\usepackage{fullpage}
\usepackage{graphicx}

\pagestyle{empty}

\begin{document}
  \section{Depth-first Search}

    \subsection{Concepts and Algorithm}

      \subsubsection{Graph Search (Traversal)}

      \emph{Graph traversal} is a list of its vertices and edges
      and an algorithm, that yields this list.

      \subsubsection{Basic Algorithm}

      \emph{Depth-first search} is defined as follows:
      \begin{enumerate}
        \item Consider a vertex $v$ that was not visited before
        \item Mark $v$ as visited
        \item Find edges $e: v \to u$ coming from $v$,
          run depth-first search from all $u$
      \end{enumerate}

      It's easy to see that the algorithm visits all vertices,
      that have a path to it from source vertex.

      This algorithm is widely run from all non-visited vertices
      sequentially. So the traversal will contain all vertices,
      ignoring the connectivity.

      \subsubsection{Visit Marks}

      For every vertiex the algorithm needs to rememnber if it is
      visited. One usually writes this information to a boolean array:
      for vertex $v$ one stores $visited[v]$, initially $false$.
      As we'll see, for every vertex it's useful to store some other
      data.

      \subsubsection{Traversal Tree}

      The algorithm visits every vertex no more than once, because as
      soon as it happens, the vertex will be marked visited.
      So the visited vertices form a tree, rooted at the source vertex.

      To build the tree implicitly, one can remember its parent at
      the start of the recursive call.

      \subsubsection{Edges Classes}

      One can classify edges based on the traversal.

      \begin{enumerate}
        \item \emph{Tree edges} are the edges connecting each vertex
          to its parent in the traversal tree. These are the edges
          used by the algorithm to find new vertices.
        \item \emph{Back edges} are the edges from vertices to their
          ancestors in the tree.
        \item \emph{Forward edges} are the edges from vertices to
          their descendants.
        \item \emph{Cross Edges} are all edge not classified before.
      \end{enumerate}

      \begin{itemize}
        \item For a directed graph, every back edge corresponds to
          a simple cycle.

        \item For an undirected graph:
          \begin{enumerate}
            \item There are no cross edges (prove it).
            \item Every forward edge $u \to v$ is also a back edge $v \to u$
            \item Every back edge $u \to v$ is also either
              forward edge $v \to u$ or tree edge $v \to u$.
          \end{enumerate}
      \end{itemize}

      \subsubsection{Inbound And Outbound Times}

      One can also add timestamps: record the order of start and end
      of vertex processing. Commonly sequential integers are used.
      Let's define inbound time as $t^{in}_v$ and outbound time as
      $t^{out}_v$.

      Note that these times form a bracket sequence:
      \begin{itemize}
        \item $t^{in}_v < t^{in}_u < t^{out}_u < t^{out}_v$, if vertex
          $v$ is an ancestor of $u$,
        \item $t^{in}_u < t^{in}_v < t^{out}_v < t^{out}_u$, if vertex
          $v$ is a descendant of $u$,
        \item $t^{in}_v < t^{out}_v < t^{in}_u < t^{out}_u$ or
          $t^{in}_u < t^{out}_u < t^{in}_v < t^{out}_v$ otherwise.
      \end{itemize}

      \subsubsection{Three Colors}

      One can use a more complicated marking system. One of them is
      commonly described as three colors system.

      Initially all vertices are white: the processing is not started.
      Immediately after the start of processing the vertex is marked
      gray. At the end of processing it is marked black.

      It is a nice way to find back edges: these are the only edges
      from gray vertex to another gray vertex. Tree edges connect
      gray vertices to white, all other edges connect gray to black.

      \subsubsection{* Implementation Notes For Trees}

      Consider the depth-first search over a tree. The only edges
      not in the traversal tree are the corresponding back edges.
      The statement above holds for an undirected tree. It also
      allows not to store visit marks. To avoid infinite recursive
      calls, it's enough not to allow going from a vertex to its parent.

    \subsection{Applications}

      \subsubsection{Checking For A Ancestor}

      \subsubsection{Connected Components}
      
      \subsubsection{Cycles}
      
      \subsubsection{Topological Sorting}

      \subsubsection{Two-Color Coloring}

      \subsubsection{Strongly Connected Components}

      \subsubsection{Bridges}

      \subsubsection{Cut Vertices}

      \subsubsection{Biconnected Components}

      \subsubsection{* Eulerian Cycle}

      \subsubsection{* Maximal Matching In A Bipartite Graph}

      \subsubsection{* Least Common Ancestor}

\end{document}
