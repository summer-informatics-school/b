\documentclass[a4paper,10pt]{article}

\usepackage[utf8]{inputenc}
\usepackage[T2A]{fontenc}
\usepackage[english,russian]{babel}

\usepackage{graphicx}
\usepackage{color}
\usepackage{indentfirst}
\usepackage{fancyhdr}
\usepackage{needspace}
\usepackage[margin=2cm]{geometry}
\usepackage{tikz}
\usetikzlibrary{arrows}

\pagestyle{fancy}
\lhead{Остовы}
\rhead{ЛКШ.2015.Август.B.Тесты}

\begin{document}
  \section{Остовы: тест}

  \begin{enumerate}
    \item
      Заполните табличку (V – количество вершин, E – количество ребер).
      \vspace{-0.5cm}
      \begin{center}
        \begin{tabular}{|p{2.5cm}|p{4cm}|p{4cm}|p{4cm}|}
          \hline
          & Алгоритм Прима & Алгоритм Прима с приоритетной очередью & Алгоритм Крускала \\
          \hline
          Время работы & & & \\
          \hline
        \end{tabular}
      \end{center}

    \item Для данного графа пометьте рёбра, которые входят во все минимальные
    остовы (x), входят хотя бы в один минимальный остов, но не во все (y),
    и которые не входят ни в один минимальный остов (z).

    \begin{center}
      \begin{tikzpicture}[auto,node distance=1.8cm,
        thick,main node/.style={circle,draw,font=\sffamily\bfseries}]
          \node[main node] (1) {g};
          \node[main node] (2) [below of=1] {e};
          \node[main node] (3) [below left of=1] {c};
          \node[main node] (4) [below right of=1] {d};
          \node[main node] (5) [below of=3] {b};
          \node[main node] (6) [below of=4] {f};
          \node[main node] (7) [below right of=5] {a};

          \path[] (3) [] edge node [] {3} (1);
          \path[] (1) [] edge node [] {4} (4);
          \path[] (3) [] edge node [] {3} (2);
          \path[] (2) [] edge node [] {2} (4);
          \path[] (2) [] edge node [] {4} (6);
          \path[] (3) [] edge node [] {2} (5);
          \path[] (5) [] edge node [] {2} (6);
          \path[] (5) [] edge node [] {2} (7);
          \path[] (4) [] edge node [] {3} (6);
          \path[] (6) [] edge node [] {2} (7);
      \end{tikzpicture}
    \end{center}

    \item Предположим, что в графе есть ребра отрицательного веса со значением
    не менее $-K$ ($K>0$). Прибавим ко всем ребрам графа положительное число $K$.
    Теперь в графе нет ребер отрицательного веса. Можно ли применить алгоритм
    Дейкстры к получившемуся графу, чтобы найти кратчайшие пути для исходного
    графа? Ответ обоснуйте.

    \vskip 3.5cm

    \item Дана СНМ на 10 элементах, реализованная с ранговой эвристикой. Изначально все
    вершины находятся в отдельных компонентах. Приведите такую последовательность операций
    вида \texttt{union(a, b)}, что после их выполнения в указанном порядке поддерево с
    корнем в $0$ будет иметь ранг $3$. Считайте, что при равенстве рангов корень второй
    компоненты (b) подвешивается к корню первой (a).

    \vskip 3.5cm

    \item Примените алгоритм Форда-Беллмана к ориентированному графу, найдите расстояния
    до всех вершин из вершины $1$, описывая текущий результат после каждой итерации.
    Порядок ребер для рассмотрения: $(1, 2)$, $(1, 3)$, $(1, 6)$, $(2, 3)$, $(2, 4)$,
    $(3, 4)$, $(6, 3)$, $(4, 5)$, $(5, 6)$.
        
      \begin{center}
        \begin{tikzpicture}[auto,node distance=1.8cm,
          thick,main node/.style={circle,draw,font=\sffamily\bfseries}]
            \node[main node] (5) [] {5};
            \node[main node] (6) [below left of=5] {6};
            \node[main node] (3) [below of=5] {3};
            \node[main node] (4) [below right of=5] {4};
            \node[main node] (1) [below left of=3] {1};
            \node[main node] (2) [below right of=3] {2};

            \path[] (6) [] edge node [] {3} (5);
            \path[] (5) [] edge node [] {1} (4);
            \path[] (6) [] edge node [] {-3} (3);
            \path[] (3) [] edge node [] {-2} (4);
            \path[] (1) [] edge node [] {9} (6);
            \path[] (3) [] edge node [] {4} (1);
            \path[] (3) [] edge node [] {5} (2);
            \path[] (4) [] edge node [] {10} (2);
            \path[] (2) [] edge node [] {2} (1);
        \end{tikzpicture}
      \end{center}

  \end{enumerate}
\end{document}
