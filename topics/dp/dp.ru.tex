\documentclass[a4paper,12pt]{article}

\usepackage[utf8]{inputenc}
\usepackage[T2A]{fontenc}
\usepackage[english,russian]{babel}

\usepackage{fullpage}
\usepackage{graphicx}
\usepackage{indentfirst}

\pagestyle{empty}

\begin{document}
  \section{Динамическое программирование}

    \subsection{Общие принципы}

      \subsubsection{Построение, оптимизация, комбинаторика}

      Формулировки задач динамического программирования чаще всего
      требует найти один из трёх результатов:

      \begin{enumerate}
        \item Построить некоторый объект с набором свойств. Например,
          найти в дереве путь, проходящий ровно по двум вершинам
          из выделенных.
        \item Оптимизировать некоторый параметр. Например, найти
          наибольшую возрастающую подпоследовательность. Здесь
          подпоследовательность~--- это объект, а её длина оптимизируется.
        \item Подсчитать количество способов построить некоторый
          объект. Например, найти количество чисел из цифр $0$ и $1$,
          делящихся на $3$.
      \end{enumerate}

      \subsubsection{Подзадачи}

      Задачи, решаемые методом динамического программирования,
      имеют основную общую особенность. Задачу в них можно сформулировать
      так, что решение можно свести к решению аналогичных задач меньшего
      размера~--- \emph{подзадач}.

      (Оптимальность для подзадачи)
      
      \subsubsection{Состояния, переходы, база}

      \subsubsection{ДП как перебор с запоминанием}

      \subsubsection{Выделение параметров, избыточность}

      (Задача о трёх кучках монет)

      (Добавить параметр, чтобы преобразовать оптимизацию в построение)

    \subsection{ДП на префиксах}

      \subsubsection{Кратчайший путь в ациклическом графе}

      \subsubsection{НВП}

      \subsubsection{НОП}
    
    \subsection{ДП по цифрам числа}

      \subsubsection{Количество чисел от 0 до n с суммой цифр k}
    
    \subsection{ДП на подотрезках}

      \subsubsection{Оптимальное перемножение матриц}

      \subsubsection{Наибольшая подпоследовательность-палиндром, подматрица-палиндром}
      \subsubsection{Казино}

    \subsection{ДП на поддеревьях}

      \subsubsection{взвешенное паросочетание в дереве}

      \subsubsection{выделение поддерева размера k}

      \subsubsection{задача о разделении дерева}

\end{document}
