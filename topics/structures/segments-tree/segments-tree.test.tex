\documentclass[a4paper,10pt]{article}

\usepackage[utf8]{inputenc}
\usepackage[T2A]{fontenc}
\usepackage[english,russian]{babel}

\usepackage{graphicx}
\usepackage{color}
\usepackage{indentfirst}
\usepackage{fancyhdr}
\usepackage{needspace}
\usepackage[margin=2cm]{geometry}
\usepackage{tikz}
\usepackage[perpage]{footmisc}

\pagestyle{fancy}
\lhead{Дерево отрезков}
\rhead{ЛКШ.2015.Август.B.Тесты}

\begin{document}
  \section{Дерево отрезков: тест, вариант 1}

  \begin{enumerate}
    \item Нарисуйте дерево отрезков для операции
      $\oplus$\footnote{Символом $\oplus$ обозначается исключающее ИЛИ:
      $0 \oplus 0 = 1 \oplus 1 = 0$, $0 \oplus 1 = 1 \oplus 0 = 1$}
      для массива двоичных чисел.
      $$[101_2, 001_2, 111_2, 110_2, 010_2, 000_2, 001_2, 110_2]$$.

      \vskip 3cm

    \item Постройте разреженную таблицу для операции $\min$
      для массива $[3, 1, 7, 2, 4]$.

      \vskip 3cm

    \item Для некоторого индексируемого с нуля восьмиэлементного
      массива построено дерево отрезков (см. рисунок).
      Поступило три запроса:
      \begin{enumerate} 
        \item увеличение на пять всех элементов с индексами из $[3;7)$,
        \item уменьшение на шесть элементов с индексами из $[0;4)$,
        \item увеличение на один всех элементов с индексами из $[4; 6)$.
      \end{enumerate}

      В кружки на рисунке впишите итоговые значения полей
      \texttt{add}\footnote{Поле для хранения значения отложенной операции, которую еще
      предстоит <<протолкнуть>>. Возможные названия: \texttt{add}, \texttt{delta}, \texttt{upd}}
      для соответствующих вершин.

      \begin{center}
        \begin{tikzpicture}[level/.style={sibling distance=40mm/#1}]
          \node [circle,draw] (1) {\vphantom{0}}
            child {node [circle, draw] (2) {\vphantom{0}}
              child {node [circle, draw] (4) {\vphantom{0}}
                child {node [circle, draw] (8) {\vphantom{0}}
                }
                child {node [circle, draw] (9) {\vphantom{0}}
                }
              }
              child {node [circle, draw] (5) {\vphantom{0}}
                child {node [circle, draw] (10) {\vphantom{0}}
                }
                child {node [circle, draw] (11) {\vphantom{0}}
                }
              }
            }
            child {node [circle, draw] (3) {\vphantom{0}}
              child {node [circle, draw] (6) {\vphantom{0}}
                child {node [circle, draw] (12) {\vphantom{0}}
                }
                child {node [circle, draw] (13) {\vphantom{0}}
                }
              }
              child {node [circle, draw] (7) {\vphantom{0}}
                child {node [circle, draw] (14) {\vphantom{0}}
                }
                child {node [circle, draw] (15) {\vphantom{0}}
                }
              }
            }
          ;
        \end{tikzpicture}
      \end{center}

    \item Для массива из тринадцати элементов, нумеруемого с $0$,
      постройте дерево отрезков, пронумеруйте его вершины с единицы.
      Разбейте полуинтервал $[3; 12)$ на минимальное число отрезков,
      соответствующих вершинам построенного дерева отрезков,
      и перечислите номера этих вершин.

      Замечание по построению: если вершине соответствует полуинтервал
      $\left[L;R\right)$, то он делится на $\left[L;M\right)$ и $\left[M;R\right)$,
      где $M=\lfloor\frac{L+R}{2}\rfloor$.
  \end{enumerate}
  \newpage

  \section{Дерево отрезков: тест, вариант 2}

  \begin{enumerate}
    \item Нарисуйте дерево отрезков для операции
      $\oplus$\footnote{Символом $\oplus$ обозначается исключающее ИЛИ:
      $0 \oplus 0 = 1 \oplus 1 = 0$, $0 \oplus 1 = 1 \oplus 0 = 1$}
      для массива двоичных чисел.
      $$[101_2, 001_2, 111_2, 110_2, 010_2, 000_2, 001_2, 110_2]$$.

      \vskip 3cm

    \item Постройте разреженную таблицу для операции $\min$
      для массива $[3, 1, 7, 2, 4]$.

      \vskip 3cm

    \item Для некоторого индексируемого с нуля восьмиэлементного
      массива построено дерево отрезков (см. рисунок).
      Поступило три запроса:
      \begin{enumerate} 
        \item присвоить пять всем элементам с индексами из $[3;7)$,
        \item присвоить шесть всем элементам с индексами из $[0;4)$,
        \item присвоить один всем элементам с индексами из $[4; 6)$.
      \end{enumerate}

      В кружки на рисунке впишите итоговые значения полей
      \texttt{upd}\footnote{Поле для хранения значения отложенной операции, которую еще
      предстоит <<протолкнуть>>. Возможные названия: \texttt{add}, \texttt{delta}, \texttt{upd}}
      для соответствующих вершин.

      \begin{center}
        \begin{tikzpicture}[level/.style={sibling distance=40mm/#1}]
          \node [circle,draw] (1) {\vphantom{0}}
            child {node [circle, draw] (2) {\vphantom{0}}
              child {node [circle, draw] (4) {\vphantom{0}}
                child {node [circle, draw] (8) {\vphantom{0}}
                }
                child {node [circle, draw] (9) {\vphantom{0}}
                }
              }
              child {node [circle, draw] (5) {\vphantom{0}}
                child {node [circle, draw] (10) {\vphantom{0}}
                }
                child {node [circle, draw] (11) {\vphantom{0}}
                }
              }
            }
            child {node [circle, draw] (3) {\vphantom{0}}
              child {node [circle, draw] (6) {\vphantom{0}}
                child {node [circle, draw] (12) {\vphantom{0}}
                }
                child {node [circle, draw] (13) {\vphantom{0}}
                }
              }
              child {node [circle, draw] (7) {\vphantom{0}}
                child {node [circle, draw] (14) {\vphantom{0}}
                }
                child {node [circle, draw] (15) {\vphantom{0}}
                }
              }
            }
          ;
        \end{tikzpicture}
      \end{center}

    \item Для массива из тринадцати элементов, нумеруемого с $0$,
      постройте дерево отрезков, пронумеруйте его вершины с единицы.
      Разбейте полуинтервал $[3; 12)$ на минимальное число отрезков,
      соответствующих вершинам построенного дерева отрезков,
      и перечислите номера этих вершин.

      Замечание по построению: если вершине соответствует полуинтервал
      $\left[L;R\right)$, то он делится на $\left[L;M\right)$ и $\left[M;R\right)$,
      где $M=\lfloor\frac{L+R}{2}\rfloor$.
  \end{enumerate}
\end{document}
