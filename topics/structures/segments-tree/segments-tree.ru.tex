\documentclass[a4paper,12pt]{article}

\usepackage[utf8]{inputenc}
\usepackage[T2A]{fontenc}
\usepackage[english,russian]{babel}

\usepackage{graphicx}
\usepackage{color}
\usepackage{indentfirst}
\usepackage{fancyhdr}
\usepackage{needspace}

\pagestyle{fancy}
\lhead{Дерево отрезков}
\rhead{ЛКШ.2014.Июль.B}

\begin{document}

  \section{Дерево отрезков}

    \subsection{RMQ, RSQ, решения offline}

    Для начала рассмотрим две классические задачи, для
    решения которых в дальнейшем и применим дерево
    отрезков.

    В задаче RSQ (Range Sum Query, запросы суммы
    на отрезках) требуется поддерживать
    последовательность $a_i$ и выполнять запросы
    двух видов:
    \begin{itemize}
      \item сообщить сумму
        $a_i + a_{i+1} + \ldots + a_j$,
      \item изменить элемент $a_i$.
    \end{itemize}

    Можно решать её <<в лоб>>, поддерживая массив и
    вычисляя сумму каждый раз с начала до конца.
    Сложности запросов в таком случае составит
    $O(n)$ и $O(1)$ соответственно.

    Можно подсчитать частичные суммы:
    $s_i = \sum_{j=1}^{i}{a_j}$. Это позволит
    вычислять сумму с $i$-го по $j$-й элемент за
    $O(1)$. Однако, изменение $i$-го элемента потребует
    пересчёта всех последующих частичных сумм, то
    есть будет работать за $O(n)$.

    Придумав два <<крайних>> решения, перейдём
    к более сбалансированным.
    
    Выберем число $k$
    и разобьём нашу последовательность на части по $k$
    элементов (лишние элементы можно будет отнести
    к последней части).

    Подсчитаем суммы
    $p_i = \sum_{j=1}^{k}{a_{k(i-1)+j}}$.
    Теперь ответ можно вычислить как сумму некоторых
    $p_i$ для тех частей, которые полностью входят
    в интервал запроса, а также $a_j$, которые
    остались непокрытыми после этого.
    Изменение элемента в данном случае повлечёт
    изменение также одного значения $p_i$.
    Так мы получаем решение с оценками времени
    $O(\sqrt{n})$, $O(1)$ (докажите).

    Аналогично можно построить решения для задачи
    о поиске минимума на отрезке, и вообще любой
    ассоциативной функции. Заметим, что при отсутствии
    возможности обращать эту функцию (как в примере
    с минимумом) частичные суммы придётся пересчитывать
    полностью.

    \subsection{Общая концепция дерева отрезков}

    Продолжим идею с разбиением последовательности
    на части. А именно, будем строить двоичное дерево
    следующим образом:
    \begin{itemize}
      \item Пусть рассматривается последовательность
        $a_0, a_1, \ldots, a_{n-1}$
      \item Корню сопоставим полуинтервал
        $\left[0, n\right)$
      \item Далее рекуррентно будем сопоставлять детям
        вершины с полуинтервалом $\left[l, r\right)$
        полуинтервалы $\left[l, m\right)$ и
        $\left[m, r\right)$, где
        $m = \left \lfloor \frac{l + r}{2} \right \rfloor$
      \item Если полуинтервал состоит из одного элемента,
        детей у вершины не будет
    \end{itemize}

    В каждой вершине запишем результат выполнения
    нужной операции "--- сумму, минимум или другую, "---
    для соответствующего полуинтервала.

    \subsection{Операция обновления элемента}

    При обновлении элемента изменяется значение
    в полуинтервале, который ему соответствует, а также
    во всех полуинтервалах, которые его содержат. Несложно
    заметить, что с точки зрения дерева это все вершины,
    лежащие на пути от соответствующей вершины до корня.

    \subsection{Операция запроса на отрезке}

    Пусть нужно вычислить ответ на полуинтервале
    $\left[a, b\right)$. Будем выполнять следующую
    рекурсивную функцию:
    \begin{itemize}
      \item Пусть текущая вершина дерева $v$ соответствует
        полуинтервалу $\left[l, r\right)$
      \item Если $\left[a, b\right)$ полностью покрывает
        $\left[l, r\right)$, полуинтервал
        $\left[l, r\right)$ нужно учесть полностью "---
        учтём его, выполнив нужную операцию (в простейшем
        случае сумму) со значением, хранящемся в вершине $v$,
        после чего завершим функцию
      \item Если $\left[a, b\right)$ не пересекается с
        $\left[l, r\right)$, полуинтервал
        $\left[l, r\right)$ учитывать не нужно "---
        нужно вернуть нейтральный элемент относительно
        операции (для суммы это ноль, для минимума это
        $+\inf$)
      \item Во всех остальных случаях запустим функцию
        рекурсивно от детей $v$ и выполним операцию для
        полученных значений
    \end{itemize}

    \subsection{Оценка времени работы}

    Докажем, что время выполнения одного запроса не
    превосходит $O(\log n)$.

    Пусть из вершины был произведён рекурсивный запуск
    описанной функции (третий случай).

    \subsection{Детали реализации}

    \subsection{Примеры}
      \subsubsection{Количество инверсий}
      \subsubsection{<<Stars>>}

    \subsection{Групповые операции}
      \subsubsection{Операция изменения на отрезке}
      \subsubsection{Проталкивание значения вниз}

    \subsection{Примеры}
      \subsubsection{???}
      \subsubsection{???}

\end{document}
